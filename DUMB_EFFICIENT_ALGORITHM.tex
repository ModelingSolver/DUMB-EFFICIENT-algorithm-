*Note après l'article*

Voici l'article sans la note dans le fichier .tex :

\documentclass{article}
\usepackage[utf8]{inputenc}
\usepackage{amsmath}
\usepackage{amsfonts}
\usepackage{amssymb}

\title{DUMB EFFICIENT: A Simple and Efficient Algorithm for Solving the Traveling Salesman Problem}
\author{OMARI Chems}

\begin{document}
\maketitle

\section*{Abstract}
The Traveling Salesman Problem (TSP) is a complex problem that has been studied for decades. We propose a simple and efficient algorithm to solve this problem, called DUMB EFFICIENT. Our approach involves dividing the problem into smaller sub-problems and solving them in a hierarchical manner.

\section{Introduction}
The Traveling Salesman Problem (TSP) is a complex combinatorial optimization problem that has been extensively studied over the years. Various algorithms have been proposed to solve TSP, including exact methods such as linear programming and approximation methods such as heuristics and metaheuristics. Despite significant progress, TSP remains a challenging problem for researchers and practitioners due to its complexity and size. We propose a simple and efficient algorithm to solve this problem, called DUMB EFFICIENT. Our approach involves dividing the problem into smaller sub-problems and solving them in a hierarchical manner.

\section{Algorithm Description}
The DUMB EFFICIENT algorithm works as follows:
\begin{enumerate}
\item Create an initial clustering of cities using DBSCAN \cite{Ester1996}.
\item Create a first cluster of density zones of cities with a maximum size of 20 (C1).
\item Divide the density clusters into sub-clusters of maximum size 20 (C2's).
\item Repeat step 3 until clusters of maximum size 20 (Cx's) are obtained.
\item Optimize the number of elements for each group of clusters.
\item Determine the medoid points using PAM \cite{Kaufman1990}.
\item Join the clusters using the medoid points.
\item Solve each cluster using the nearest neighbor algorithm.
\end{enumerate}

\section{Simulation Results}
We have performed simulations to evaluate the performance of the DUMB EFFICIENT algorithm. The results show that:
\begin{itemize}
\item For problems of size between 1000 and 5000 cities, the DUMB EFFICIENT algorithm obtains high-quality results, with an accuracy greater than 90%.
\item The time complexity of the algorithm is linear, making it very efficient for large-scale problems.
\end{itemize}

\section{Conclusion}
The DUMB EFFICIENT algorithm is a simple and efficient approach to solving the Traveling Salesman Problem. It is based on a hierarchical modeling and uses clustering and optimization techniques to obtain high-quality results.

\bibliographystyle{plain}
\bibliography{references}

\begin{thebibliography}{9}
\bibitem{Ester1996} Ester, M., Kriegel, H. P., Sander, J., & Xu, X. (1996). A density-based algorithm for discovering clusters in large spatial databases with noise. Proceedings of the 2nd International Conference on Knowledge Discovery and Data Mining, 226-231.
\bibitem{Kaufman1990} Kaufman, L., & Rousseeuw, P. J. (1990). Finding groups in data: An introduction to cluster analysis. Wiley-Interscience.
\end{thebibliography}

\end{document}

*Note*
Alpha version, détails to come. Further détails on résolution methods, implémentation proposal, explanatory details of simulations, and extended comparison to other algorithms will be added in future versions.
